O Cartão Amigo dos Museus de Portugal é um cartão anual que oferece entrada gratuita e ilimitada na Rede Portuguesa de Museus composta por 156 museus nacionais,bem como descontos em eventos selecionados de várias salas de espetáculo aderentes à iniciativa.

O departamento de marketing da Direcção geral do Património Cultural identificou como prioritário o desenvolvimento de uma aplicação para a compra de bilhetes para os museus e salas de espetáculo nacionais dedicada aos aderentes dos Cartões Amigo.

Os cartões têm a validade de 12 meses e podem ser de 3 tipos\+: Cartão Amigo Uni -\/ com o custo de 32.\+45€/ano reservado a estudantes e professores universitários; Cartão Amigo Silver -\/ com o custo de 30€/ano dedicado a cidadãos com mais de 65anos; e o Cartão Amigo Individual -\/ com o custo de 54.\+90€/ano para os restantes associados. Para a emissão do cartão, será necessária a informação sobre nome, data de nascimento, morada e contacto móvel do associado, bem como a data de aquisição do cartão.

Independentemente do tipo de cartão, os detentores de Cartão Amigo dos Museus de Portugal terão sempre um desconto de 25\% em eventos selecionados que ocorram em salas de espetáculo aderentes à iniciativa.

Ainda na recente estratégia de marketing, foi decidido avançar com a opção de permitir que os detentores do cartão Silver assistam sem qualquer custo adicional a eventos a realizar-\/se num museu ou sala de espetáculos abrangida pela Rede e pertencente à área de residência do associado, caso, faltando menos do que 8 horas para o início do evento ainda não tenham sido vendido smais do que 50\% dos bilhetes disponíveis para o mesmo. Nesse caso, os detentores de cartão Silver irão receber uma mensagem com esta informação e com toda a informação necessária para reservar bilhete para o referido evento.

O sistema deve incluir e gerir informação relativa aos museus da Rede Portuguesa de Museus, sobre os detentores dos Cartões Amigo, sobre os tipos de cartão, bem como dos eventos promovidos pelas instituições aderentes à iniciativa. Deve ainda permitir reservar bilhetes para as iniciativas promovidas por estas instituições.

Implemente também outras funcionalidades que considere relevantes, para além dos requisitos globais enunciados na página inicial. 